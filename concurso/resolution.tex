\documentclass{article}
\usepackage{amsmath}
\usepackage{amsfonts}
\usepackage{amssymb}


\begin{document}
\section*{CONHECIMENTOS ESPECÍFICOS - IFSC junho 2023}
\section*{Q21}
Suponha uma expedição espacial em direção ao sistema planetário mais próximo do
nosso, Alpha Centauri, que está localizado a uma distância média de 4,22 anos-luz da Terra. Se a
expedição viajar a uma velocidade correspondente a 60\% da velocidade da luz, qual seria,
aproximadamente, a diferença, em anos, entre o tempo medido pelos astronautas durante a viagem
e o tempo medido na Terra?

\begin{enumerate}
\item A) 0,84 anos.
\item B) 1,06 anos.
\item \textbf{\Large C) 1,41 anos.}
\item D) 1,69 anos.
\item E) 2,81 anos.
\end{enumerate}

\textbf{RESOLUCAO}

A fórmula para a dilatação do tempo é:

\[ \Delta t' = \frac{\Delta t}{\sqrt{1 - \frac{v^2}{c^2}}} \]

Onde:
\begin{itemize}
\item \( \Delta t' \) é o intervalo de tempo medido pelos astronautas na nave.
\item \( \Delta t \) é o intervalo de tempo medido na Terra.
\item \( v \) é a velocidade da nave (0,6 vezes a velocidade da luz, ou \( 0,6c \)).
\item \( c \) é a velocidade da luz.
\end{itemize}
Nesse caso, a distância até Alpha Centauri é de 4,22 anos-luz e a velocidade da nave é \( 0,6c \).

Primeiro, vamos calcular o intervalo de tempo medido ($\Delta t$) pelo observador na terra:

\begin{equation}
    0.6c = \frac{4,22 \textrm{ anos-luz}}{\Delta t} 
\end{equation}
    
\begin{equation} 
    \Delta t = \frac{4,22}{0,6c} = 7,033 \textrm{ anos}
\end{equation}

Agora, vamos calcular o intervalo de tempo medido pelos astronautas na nave:

\begin{equation} 
\Delta t = \gamma\Delta t'
\end{equation}

\begin{equation}
    \gamma = \frac{1}{1-\frac{v^2}{c^2}} = \frac{1}{1-(0.6)^2} = 1.25
\end{equation}

Agora, usamos a fórmula da dilatação do tempo para encontrar a diferença de tempo entre os astronautas e a Terra:

\begin{equation} 
    7,033 = 1.25\Delta t' \rightarrow \Delta t' = \frac{7,03}{1.25} = 5.624 \textrm{anos}
\end{equation}

Portanto, a diferença aproximada entre o tempo medido pelos astronautas durante a viagem e o tempo medido na 
Terra \'e 7,033 anos - 5.624 anos = 1.406 anos. \textbf{RESPOSTA C. 1.41 ANOS}
    
\end{document}